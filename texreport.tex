% CSC 300: Professional Responsibilities Dr. Clark Turner

% Two Column Format
\documentclass[11pt]{article}
%this allows us to specify sections to be single or multi column so that things
%like title page and table of contents are single column
\usepackage{multicol}
\usepackage{indentfirst}
\usepackage[english]{babel}

\usepackage{setspace} 

\usepackage{url}

%%% PAGE DIMENSIONS
\usepackage{geometry} % to change the page dimensions 
\geometry{letterpaper}

\begin{document}

\title{\vfill Peek-a-Boo \\ \large Tor vs Iran}
\author{ 
  By Orion Miller\vspace{10pt} \\
  CSC 300: Professional Responsibilities \vspace{10pt} \\
  Dr. Clark Turner\vspace{10pt} \\
}
%\date{October 22, 2010} %Or use \today for today's Date
\date{\today}

\maketitle

\vfill  %in combinaion with \newpage this forces the abstract to the bottom of
\begin{abstract} 
  DA DA DA
\end{abstract}

\thispagestyle{empty} %remove page number from title page 

\newpage

\thispagestyle{empty}  %Remove page number from TOC 

\tableofcontents

\newpage

%end the 1 column format


%start 2 column format
\begin{multicols}{2}
%Start numbering first page of content as page 1
\setcounter{page}{1}
%%%%%%%%%%%%%%%%%%%%
%%% Known Facts  %%%
%%%%%%%%%%%%%%%%%%%%
\section{Facts} 

\subsection{Iran's Presidential Election}

In Iran, the sparks of political unrest began to fly in the summer of 2009.
Friction started with the highly anticipated outcome of the Iran's race for
presidency. Elections took place on June 12th. To the surprise of many, Mahmoud
Ahmadinajad won with a lanslide victory of well above 60\% of the vote
\cite{TheIranianVote, IranianElectionResultsByProvince}, making this
Ahmadinjad's second term as the President of Iran.

\subsection{The Response}

Iranian citizens and many others were outraged with the results of the election.
Laura Secor, from the New Yorker, succinctly sumed up the voice of the people by
stating \begin{quotation} there can be no question that on June 12, 2009, the Iranian
presidential election was stolen. \cite{TheIranianVote}\end{quotation} Citizens
swiftly began to vocalize their opinions.

\subsection{Protests Begin}

Protests began on June 13 and picked up force not only offline but online as
well. By the fifth day of protests, hundreds of thousands of demonstrators took to
the streets of Tehran in opposition of the election's outcome
\cite{IranProtestsFifthDayOfUnrest}. At the same time, ``the
Twitterverse\footnote{Twitterverse is a name referring everything relating to
the social network Twitter.} exploded with tweets from people who weren't having
it, both in English and in Farsi.'' \cite{WhyTwitterIsTheMedium}

\subsection{Iran Tries to Stop the Protests}

Iran's government tried to douse out the dissent. The government's approach was
to have an all around black out of information and organization. They sent
Revolutionary Gaurds into the streets, cut off the Internet, and shut down
cell-phone communications\cite{TheIranianVote}. The government didn't stop
there, they began to specifically arresting, ``at least 500 activitsts,
opposition figures, journalists.'' \cite{IranProtestsFifthDayOfUnrest}.  Some
locations for demonstration a possibly threat to life.  One such location was
the Interior Ministry\footnote{The Interior Ministry was the part of the
government that tallied the votes of the election}. The government publicly
announced that the police had orders to shoot anyone approaching the Interior
Ministry \cite{TheIranianVote}. Online, the government filters their internet
traffic and for periods of times block social networks like
Twitter\footnote{Twitter is a microblogging service were anyone can read, write,
and share messages, up to 140 characters long.\cite{WhatIsTwitter}}
\cite{IranBlocksFacebookTwitter}.

%%TWITTER FOOTNOTE - add source from "What is Twitter, a social network or a
%% News media
\subsection{The Role of Social Networks}

Twitter began to prove itself an invaluabale tool for protesters. It source of
information and played a vital role during the protestes because traditional
means of sharing informations such as newspapers were breaking down. Iran,
``while the front pages of Iranian newspapers were full of blank space where
censors had whited-out news stories, Twitter was delivering information from
street level, in real time.''\cite{WhyTwitterIsTheMedium}

\subsection{Citizens use Tor to Bypass Censorship}

With the government filtering out vital lifelines of information and means of
organization citizens began to find alternatives to connect to Twitter. One of
the tools that proved useful was Tor.

\subsection{What is Tor}

Tor is a tool and network that provides users a means way of connecting with
websites encrypted and anonymously. From Tor's website under a page about the
the 'Users of Tor' they state that, ``Tor was origianlly designed, implemented
and deployed as a third-generation onion routing project of the Naval Research
Laboratory... Today, it is used every day for a wide variety of purposes by the
military, journalists, law enofrcement officers, activists, and many others.

\subsection{Iran Blocks Tor}

Fast forward two years and Iran is still under the same regime of Ahmadinajad.
On September 14, 2011 the Iranian government added a filter rule to their
firewall and blocked Tor traffic despite Tors main focused intent on preventing
such things from ever happening\cite{IranBlocksTorSameDayFix}.  Later that day
the Tor project released a patch which circumvent Iran's firewall filters and
suggesteds users of to to upgrade to version Tor 0.2.2.33 or Tor
0.2.3.4-alpha\cite{IranBlocksTorSameDayFix}.


%%twitter's vital importance

%%twitter and other cites were being censored



%%%%%%%%%%%%%%%%%%%%%%%%%
%%% Research Question %%%
%%%%%%%%%%%%%%%%%%%%%%%%%
\section{Research Question} 
Was it ethical for the Tor project to release a same day patch specifically
enabling Iranians to bypass Iran's firewall when the government explicitely
filtered Tor traffic starting September 14, 2011.

\subsection{Importance}

The Internet, blogs, and social networks have proven to play a critical role in
Iranian political discorse at least for the past 3 years. With Iran's government
strive to squander any form of freedom of expression or press by force both on-
and off-line has only hightened the need for a means to safely, privately, and
without restriction to communicate online. The Tor project strives to make Tor
able to provide such services to the world and in this case specifically to aid
Iranians. There project and focused reason to patch their software could have
grave implications for Iranian users.

%%%%%%%%%%%%%%%%%%%%%%%%%
%%% Extant Arguments from External Sources %%%
%%%%%%%%%%%%%%%%%%%%%%%%%
\section{Extant Arguments} 

There have virtually been no published articles specifically finding the Tor's
patch in response to Iran's Tor block ethical or not.

\subsection{Arguments For Tor}

Tor in general has been considerd for good of humanity, giving users right to
privacy while using the Internet \cite{TorCreatesSaferInternet, TorLastHope}.
The Tor project has even been given The Free Software Foundation Award of 2010
for 
\begin{quotation}Tor has enabled roughly 36 million people around the world
  to experience freedom of access and expression on the Internet while keeping
  them in control of their privacy and anonymity. Its network has proved pivotal 
  in dissident movements in both Iran and more recently Egypt.\cite{FreeSoftwareAwards}
\end{quotation}

%%people don't find tor EVIL

%%3rd parties think the censorship is bad

\subsection{Arguments Against Tor}

The Iranian government has and currently firmly believes that it is necessary to
censor the Internet traffic coming in and out of the country. It has been
announced one reason why they feel that is right to censor is because they feel
they need to stamp out ``immoral and illegal'' content
\cite{CensorshipFearsRise}.

In specific regard it is probably safe to assume that the Iranian government
finds Tor to be unethical since it allows people to bypass what they consider to
be necessary Internet filters and since the Iranian government has specifically
blocked Tor\cite{IranBlocksTorSameDayFix}.


\subsection{Arguments Summary}

To summarize, it is a common belief that Tor is ethical because it allows to
people to search anonymously. Iranian government finds Tor unethical and has
taken measure filter Tor traffic.

%%what tor can enable

%%government believe it is right


%%%%%%%%%%%%%%%%
%%% Analysis %%%
%%%%%%%%%%%%%%%%
\section{Analysis}

\subsection{Why the SE Code of Ethics Applies} 

The Software Engineering Code of Ethics states that, ``Software engineers are
those who contribute by direct participation or by teaching, to the analysis,
specification, design, development, certification, maintenance and testing of
software systems.'' \cite{SE:CodeOfEthics} The Tor Project is a 501(c)(3)
non-profit based in the United States that makes internet anonymity
software.\cite{Tor:FAQ, Tor:CorePeople} The Tor Project publicizes that, ``The
Tor software is a program you can run on your computer that helps keep you safe
on the Internet.''\cite{Tor:FAQ} Because the Tor Project creates and distributes
software this means they, ``contribute by direct participation ...  to the
analysis, specification, design, development, ... , maintenance and testing of
software systems'' and thus are Software Engineers means that The Tor Project
falls under the requirements of the Software Engineering Code of
Ethics.\cite{Tor:FAQ, Tor:Overview} 


\subsection{SE Code 3.03}

\fbox{ 
  \parbox{0.8\linewidth}{ 
    3.03. ... Address \textbf{ethical}, \textbf{economic}, (and)
    \textbf{cultural}, ... issues related to \textbf{work projects}.
}}
\newline

To bring this ethical argument into domain of the Tor Project
aiding Iranian dicicdents a few modifications to the code need to be made. The
\textbf{ethical} issue in this case is the issue of \textbf{freedom of
expression}. The \textbf{economic} issues in this case is the issue of
\textbf{Tor software's financial cost} (i.e. the financial cost on the users of
the software). The \textbf{cultural} issues in this case is refering to
\textbf{Iran's oppression of online dissidents}. The phrase \textbf{work projects} in
this case referes to \textbf{Tor software}.

%The Tor software is a \textbf{work project} that has identified, defined, and
%addressed is to directly project directly concerned its software with the one of
%its hopes to provide, ``Journalists use Tor to communicate more safely with
%whistleblowers and dissidents.

\subsubsection{Applying SE Code 3.03 to the Tor Project}

\fbox{
  \parbox{0.8\linewidth}{
    3.03. ... Address \textbf{freedom of expression},
    \textbf{Tor's financial cost}, (and) \textbf{Iran's oppression of dissidents},
    ... issues related to \textbf{Tor software}.  
}}
\newline

Our question now can be broken up into four parts. First, does Tor software
promote freedom of expression in Iran? Second, what is the financial cost for
Tor users?  Third, does Tor software mitigate Iran's oppression of dissidents?
Lastly, is Tor ethical under SE code 3.03.

\subsubsection{The state of Freedom of Expression in Iran}
Before we investigate wether Tor addresses freedom of expression in Iran we need
to examine what the current sate of freedom of expression is in Iran. 

\paragraph{Freedom of Expression Definition}

For analysis of this paper we will need define `freedom of expression' somewhat
broadly. It will be synonymsous with freedom of expression, speech, opinion, and
press.  This needs to be the case because of the nature that the Internet,
social networks and blogs have played in this case. The foundation of this
redefintion of `freedom of expression' stems from 
\begin{quotation} 
  A free press is defined not only by the media's freedom to say what they want
  but also by media consumers' freedom to get the information they need.
  \cite{PublicAttitudeTowardFreedomPress, ComplexRoadToHappiness}
\end{quotation}

In addition we need note that the terms `media' and `media consumers' from the
quote above in our case need to include to the people who post content and read
content from blogs, Twitter, social networks, and other forms of communication
online since the importance of these forms of medium have played in Iran but in
addition to that for example Twitter has been shown to show accurate news
events.  \cite{Twitter:BreakingNewsDetection, Twitter:IdentificationLiveEvents,
Twitter:MeasuringInfluence}


\paragraph{Universal Declaration of Human Rights}

We should examine Iran's relationship to the United Nations to temper our
analysis of there actions in how they cesnor their citizens. Iran is and has
been a member of the United Nations since 1945. \cite{UN:IranBecameMember,
UN:IranActiveMember} Under the United Nations `The Universal Declaration of
Human Rights' it states that ``...  Member States have pledged themselves to
achieve, in co-operation with the United Nations, the promotion of universal
respect for and observance of human rights and fundamental
freedoms.''\cite{UniversalDeclerationOfHumanRights}. It appears that under this
pledge Member States should strive to uphold `The Universal Declaration of Human
Rights' preamble and articles, Article 19 states

\begin{quotation} Everyone has the right to freedom of opinion and expression;
  this right include freedom to hold opinions without interference and to seek,
  receive and impart information and ideas through any media and regardless of
  frontiers.  \cite{UniversalDeclerationOfHumanRights}

\end{quotation}


\paragraph{Iranian Government's Censorship}

%statistics on websites shutdown
%considered how bad it is
%reasons they claim they shutdown
%reasons why other people believe it is being shut down

\subsubsection{Does Tor promote freedom of expression?}

It is important to take note of not only the actions that taken or in this case
what Tor software has done and provides users but the driving ethos behind the
project by the projects maintainers, The Tor Project.

\paragraph{Tor Projects's Ethos}

The organization that actively maintains Tor takes their software and mission
versy seriously. It can be seen in The Tor Project's Mission Statement


\begin{quotation}
  
  The Corporation is organized and shall be operated exclusively for scientific,
  charitable and educational purposes within the meaning of section 501(c)(3) of
  the Internal Revenue Code, including (a) to develop, improve and distribute
  free, publicly available tools and programs that promote free speech, free
  expression, civic engagement and privacy rights online; (b) to conduct
  scientific research regarding, and to promote the use of and knowledge about,
  such tools, programs and related issues around the world; (c) to educate the
  general public around the world about privacy rights and anonymity issues
  connected to Internet use.
  
\cite{TOR:Sponsorship}

``The Tor project is based on the belief that anonymity is not just a good idea
some of the time — it is a requirement for a free and functioning
society.''\cite{Tor:Users}

\end{quotation}



%tor project's claim of goals
%

\paragraph{Tor and Free Speech}

%tor unrestrics access to internet
%tor quote on helping out people
%EFF relay project
%EFF recommondation of using Tor


\subsubsection{What is the financial cost of Tor?}

We should examine the financial costs of Tor not only in the direct payment of
the software but it's indirect costs by being open-source.

\paragraph{Tor in Regards to Beer}


%tor in currency doesn't cost anything
%quote about some dissendents not having much money

\paragraph{Tor in Regards to Speech}

%tor is open source
%financial issues indirectly
%this allows them to view the source and make sure people are not spying on them
%if there are problems localized to Iran they can use Tor to 

\subsubsection{The State of Physical Oppression of Dissidents}

\paragraph{Protesters Jailed}

%number total protesters jailed
%places off limits for protesting

\paragraph{Bloggers Killed}

%specific blogger killed
%addressed other bloggers jailed and killed

\subsubsection{Does Tor software mitigate Iran's physical oppression of
dissidents?}

% talk about tor hiding anonymity
% talk about how it does make it more challenging for traffic analaysis
% no node can know about other nodes

\subsubsection{Is Tor ethical under SE code 3.03?}


\subsection{SE Code 1.07} 

\fbox{
  \parbox{0.8\linewidth}{
    1.07. Consider issues ...  and \textbf{other factors} that can
    \textbf{diminish access} to the \textbf{benefits} of \textbf{software}.
}}
\newline

By the the Tor project releasing a patch on September 14 the same day that Iran
added filters to block out Tor traffic in their firewall meant that they
considered that the factors of the Iranian government where diminishing access
to the benefits of their software.

To bring this ethical argument into the domain of the research question a few
modifications to the code need to meade. The phrase \textbf{other factors} in
this case is \textbf{Iran's firewall block of Tor}. The phrase \textbf{diminish
access} in this case \textbf{prevents access}. The word \textbf{benefits} in
this case is the \textbf{uncensored Internet}. The word \textbf{software} in this
case is \textbf{Tor}.

\subsubsection{Applying SE Code 1.07 to the Tor Project}

\fbox{ 
  \parbox{0.8\linewidth}{ 
    1.07. Consider the issue of ... \textbf{Iran's firewall block of Tor} that 
    \textbf{preventents access} to the \textbf{uncensored Internet} from 
    \textbf{Tor}.
}}
\newline

Our question now can be broken down into three parts. First, how did Tor
circumvent Iran's firewall? Second, how did the Tor software patch affect
uncensored Internet access?  Lastly was Tor ethical under SE Code 1.07?

\subsubsection{Iran's Firewall}

\subsubsection{How did Tor circumvent Iran's firewall?}

\paragraph{How Tor Works}

\paragraph{Dousing the Firewall}

\subsubsection{How did the Tor software patch affect uncensored Internet
access?}

\subsubsection{Lastly was Tor ethical under SE Code 1.07?}

\subsubsection{Rawlsian Justice Principle Analysis}
%%can another ethical system apply

%%preamble ''public interest''
%%the Rawlsian Justice principle gives us another means of critical analysis of
%%what would the public interest be in this case

\paragraph{Applying Rawlsian}

%%citizens are the majority has the most to gain and government who has most power

%%%%%%%%%%%%%%%%%%
%%% Conclusion %%%
%%%%%%%%%%%%%%%%%%
\section{Conclusion} The conclusion is a summary of your entire anal- ysis. It
should reiterate the answer your audience has been forming while reading your
analysis. New information should never be introduced in your conclusion.
\cite{texTemp}

%end the two column format
\end{multicols} \newpage

%cite all the references from the bibtex you haven't explicitly cited
\nocite{*}

\bibliographystyle{IEEEannot}

\bibliography{texreport}

\end{document}
