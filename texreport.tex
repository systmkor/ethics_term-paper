% CSC 300: Professional Responsibilities Dr. Clark Turner

% Two Column Format
\documentclass[11pt]{article}
%this allows us to specify sections to be single or multi column so that things
%like title page and table of contents are single column
\usepackage{multicol}

\usepackage{setspace} 

\usepackage{url}

%%% PAGE DIMENSIONS
\usepackage{geometry} % to change the page dimensions 
\geometry{letterpaper}

\begin{document}

\title{\vfill Ethical Government Computer Break-ins for Human Rights} %\vfill
gives us the black space at the top of the page \author{ By Orion
Miller\vspace{10pt} \\ CSC 300: Professional Responsibilities\vspace{10pt} \\
Dr. Clark Turner\vspace{10pt} \\ }
%\date{October 22, 2010} %Or use \today for today's Date
\date{\today}

\maketitle

\vfill  %in combinaion with \newpage this forces the abstract to the bottom of
\begin{abstract} DA DA DA\end{abstract}

\thispagestyle{empty} %remove page number from title page 

\newpage


%Create a table of contents with all headings of level 3 and above.
%http://en.wikibooks.org/wiki/LaTeX/Document_Structure#Table_of_contents has
%info on customizing the table of contents

\thispagestyle{empty}  %Remove page number from TOC 

\tableofcontents

\newpage

%end the 1 column format


%start 2 column format
\begin{multicols}{2}
%Start numbering first page of content as page 1
\setcounter{page}{1}
%%%%%%%%%%%%%%%%%%%%
%%% Known Facts  %%%
%%%%%%%%%%%%%%%%%%%%
\section{Facts} 

\subsection{Iran's Presidential Election}

In Iran, the sparks of political unrest began to fly in the summer of 2009.
Friction started with the highly anticipated outcome the Iran's race for
presidency. Elections took place on June 12th. To the surprise of many Mahmoud
Ahmadinajad won with a lanslide victory of well above 60\% of the vote
\cite{TheIranianVote, IranianElectionResultsByProvince}, making this
Ahmadinjad's second term as the President of Iran.

\subsection{The Response}

Iranian citizens and many others were outraged with the results of the election.
Laura Secor, from the New York, succinctly sumed up the voice of the people by
stating \begin{quotation} There can be no question that June 12, 2009, Iranian
presidential election was stolen. \cite{TheIranianVote}\end{quotation} Citizens
swiftly began to vocalize their opinions.

\subsection{Protests Begin}

Protests began on June 13 and picked up force not only off-line but online as
well. By the fifth day of protest hundreds of thousands of demonstrators took to
the streets of Tehran in opposition of the election's outcome
\cite{IranProtestsFifthDayOfUnrest}. At the same time, ``the
Twitterverse\footnote{Twitterverse is a name referring everything relating to
the social network Twitter.} exploded with tweets from people who weren't having
it, both in English and in Farsi.'' \cite{WhyTwitterIsTheMedium}

\subsection{Iran Tries to Stop the Protests}

Iran's government tried to douse out the dissent. The government's approach was
to have an all around black out of information and organization. They sent
Revolutionary Gaurds into the streets, cut off the Internet, and shut down
cell-phone communications\cite{TheIranianVote}. The government didn't stop
there, they began to specifically arresting, ``at least 500 activitsts,
opposition figures, journalists.'' \cite{IranProtestsFifthDayOfUnrest}.  Some
locations for demonstration a possibly threat to life.  One such location was
the Interior Ministry\footnote{The Interior Ministry was the part of the
government that tallied the votes of the election}. The government publicly
announced that the police had orders to shoot anyone approaching the Interior
Ministry \cite{TheIranianVote}. Online, the government to filter their internet
traffic and for periods of times block social networks like
Twitter\footnote{Twitter is a microblogging service were anyone can read, write,
and share messages, up to 140 characters long.\cite{WhatIsTwitter}}
\cite{IranBlocksFacebookTwitter}.

%%TWITTER FOOTNOTE - add source from "What is Twitter, a social network or a
%% News media
\subsection{The Role of Social Networks}

Twitter began to prove itself an invaluabale tool for protesters. It source of
information and played a vital role during the protestes because traditional
means of sharing informations such as newspapers were breaking down. Iran,
``while the front pages of Iranian newspapers were full of blank space where
censors had whited-out news stories, Twitter was delivering information from
street level, in real time.''\cite{WhyTwitterIsTheMedium}

\subsection{Citizens use Tor to Bypass Censorship}

With the government filtering out vital lifelines of information and means of
organization citizens began to find alternatives to connect to Twitter. One of
the tools that proved useful was Tor.

\subsection{What is Tor}

Tor is a tool and network that provides users a means way of connecting with
websites encrypted and anonymously. From Tor's website under a page about the
the 'Users of Tor' they state that, ``Tor was origianlly designed, implemented
and deployed as a third-generation onion routing project of the Naval Research
Laboratory... Today, it is used every day for a wide variety of purposes by the
military, journalists, law enofrcement officers, activists, and many others.

\subsection{Iran Blocks Tor}

Fast forward two years and Iran is still under the same regime of Ahmadinajad.
On September 14, 2011 the Iranian government added a filter rule to their
firewall and blocked Tor traffic despite Tors main focused intent on preventing
such things from ever happening\cite{IranBlocksTorSameDayFix}.  Later that day
the Tor project released a patch which circumvent Iran's firewall filters and
suggesteds users of to to upgrade to version Tor 0.2.2.33 or Tor
0.2.3.4-alpha\cite{IranBlocksTorSameDayFix}.


%%twitter's vital importance

%%twitter and other cites were being censored



%%%%%%%%%%%%%%%%%%%%%%%%%
%%% Research Question %%%
%%%%%%%%%%%%%%%%%%%%%%%%%
\section{Research Question} 
Was it ethical for the Tor project to release a same day patch specifically
enabling Iranians to bypass Iran's firewall when the government explicitely
filtered Tor traffic starting September 14, 2011.

\subsection{Importance}

The Internet, blogs, and social networks have proven to play a critical role in
Iranian political discorse at least for the past 3 years. With Iran's government
strive to squander any form of freedom of expression or press by force both on-
and off-line has only hightened the need for a means to safely, privately, and
without restriction to communicate online. The Tor project strives to make Tor
able to provide such services to the world and in this case specifically to aid
Iranians. There project and focused reason to patch their software could have
grave implications for Iranian users.

%%%%%%%%%%%%%%%%%%%%%%%%%
%%% Extant Arguments from External Sources %%%
%%%%%%%%%%%%%%%%%%%%%%%%%
\section{Extant Arguments} 

There have virtually been no published articles specifically finding the Tor's
patch in response to Iran's Tor block ethical or not.

\subsection{Arguments For Tor}

Tor in general has been considerd for good of humanity, giving users right to
privacy while using the Internet \cite{TorCreatesSaferInternet, TorLastHope}.
The Tor project has even been given The Free Software Foundation Award of 2010
for \begin{quotation}Tor has enabled roughly 36 million people around the world
  to experience freedom of access and expression on the Internet while keeping
  them in control of their privacy and anonymity. Its network has proved pivotal
in dissident movements in both Iran and more recently
Egypt.\cite{FreeSoftwareAwards}\end{quotation}

%%people don't find tor EVIL

%%3rd parties think the censorship is bad

\subsection{Arguments Against Tor}

The Iranian government has and currently firmly believes that it is necessary to
censor the Internet traffic coming in and out of the country. It has been
announced one reason why they feel that is right to censor is because they feel
they need to stamp out ``immoral and illegal'' content
\cite{CensorshipFearsRise}.

In specific regard it is probably safe to assume that the Iranian government
finds Tor to be unethical since it allows people to bypass what they consider to
be necessary Internet filters and since the Iranian government has specifically
blocked Tor\cite{IranBlocksTorSameDayFix}.


\subsection{Arguments Summary}

To summarize, it is a common belief that Tor is ethical because it allows to
people to search anonymously. Iranian government finds Tor unethical and has
taken measure filter Tor traffic.

%%what tor can enable

%%government believe it is right


%%%%%%%%%%%%%%%%
%%% Analysis %%%
%%%%%%%%%%%%%%%%
\section{Analysis}

\subsection{Why the SE Code of Ethics Applies} 

The Software Engineering Code of Ethics states that, ``Software engineers are
those who contribute by direct participation or by teaching, to the analysis,
specification, design, development, certification, maintenance and testing of
software systems.'' \cite{SE:CodeOfEthics} The Tor Project is a 501(c)(3)
non-profit based in the United States that makes internet anonymity
software.\cite{Tor:FAQ, Tor:CorePeople} The Tor Project publicizes that, ``The
Tor software is a program you can run on your computer that helps keep you safe
on the Internet.''\cite{Tor:FAQ} Because the Tor Project creates and distributes
software this means they, ``contribute by direct participation ...  to the
analysis, specification, design, development, ... , maintenance and testing of
software systems'' and thus are Software Engineers means that The Tor Project
falls under the requirements of the Software Engineering Code of
Ethics.\cite{Tor:FAQ, Tor:Overview} 


\subsection{SE Code 3.03}

\fbox{
  \parbox{2.75in}{
    3.03. Identify, define and address \textbf{ethical}, \textbf{economic},
    (and) \textbf{cultural}, ... issues related to \textbf{work projects}. 
}}\vspace{0.25in}

To bring this ethical argument into domain of the Tor Project
aiding Iranian dicicdents a few modifications to the code need to be made. The
\textbf{ethical} issue in this case is the issue of \textbf{freedom of
expression}. The \textbf{economic} issues in this case is the issue of
\textbf{Tor software's financial cost} (i.e. the financial cost on the users of
the software). The \textbf{cultural} issues in this case is refering to
\textbf{Iran's oppression of online dissidents}. The phrase \textbf{work projects} in
this case referes to \textbf{Tor software}.

The Tor software is a \textbf{work project} that has identified, defined, and
addressed is to directly project directly concerned its software with the one of
its hopes to provide, ``Journalists use Tor to communicate more safely with
whistleblowers and dissidents.

%ethical issues - freedom of expression
%economic issues - cost of software
%cultural issues - jail & death for freedom of expression

\subsubsection{Applying SE Code 3.03 to the Tor Project}

\fbox{
  \parbox{2.75in}{
    3.03. Identify, define and address \textbf{freedom of expression},
    \textbf{Tor's financial cost}, (and) \textbf{Iran's oppression of online
    dissidents}, ... issues related to \textbf{Tor software}. 
}}\vspace{0.25in}


\subsection{SE Code 1.07} 

\fbox{
  \parbox{2.75in}{
    1.07. Consider issues ...  and \textbf{other factors} that can
    \textbf{diminish access} to the \textbf{benefits} of \textbf{software}.
}}\vspace{0.25in}

\subsubsection{Applying SE Code 1.07 to the Tor Project}



%%%%%%%%%%%%%%%%%%
%%% Conclusion %%%
%%%%%%%%%%%%%%%%%%
\section{Conclusion} The conclusion is a summary of your entire anal- ysis. It
should reiterate the answer your audience has been forming while reading your
analysis. New information should never be introduced in your conclusion.
\cite{texTemp}

%end the two column format
\end{multicols} \newpage

%cite all the references from the bibtex you haven't explicitly cited
\nocite{*}

\bibliographystyle{IEEEannot}

\bibliography{texreport}

\end{document}
