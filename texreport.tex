% CSC 300: Professional Responsibilities Dr. Clark Turner

% Two Column Format
\documentclass[11pt]{article}
%this allows us to specify sections to be single or multi column so that things
%like title page and table of contents are single column
\usepackage{multicol}

\usepackage{setspace} \usepackage{url}

%%% PAGE DIMENSIONS
\usepackage{geometry} % to change the page dimensions \geometry{letterpaper}

\begin{document}

\title{\vfill Ethical Government Computer Break-ins for Human Rights} %\vfill
gives us the black space at the top of the page \author{ By Orion
  Miller\vspace{10pt} \\ CSC 300: Professional Responsibilities\vspace{10pt} \\
  Dr. Clark Turner\vspace{10pt} \\ }
%\date{October 22, 2010} %Or use \today for today's Date
\date{\today}

\maketitle

\vfill  %in combinaion with \newpage this forces the abstract to the bottom of
\begin{abstract} DA DA DA\end{abstract}

\thispagestyle{empty} %remove page number from title page \newpage


%Create a table of contents with all headings of level 3 and above.
%http://en.wikibooks.org/wiki/LaTeX/Document_Structure#Table_of_contents has
%info on customizing the table of contents
\thispagestyle{empty}  %Remove page number from TOC \tableofcontents

\newpage

%end the 1 column format


%start 2 column format
\begin{multicols}{2}
%Start numbering first page of content as page 1
\setcounter{page}{1}
%%%%%%%%%%%%%%%%%%%%
%%% Known Facts  %%%
%%%%%%%%%%%%%%%%%%%%
\section{Facts} 

In Iran, the sparks of political unrest began to fly in the summer of 2009.
Friction started with the highly anticipated outcome the Iran's race for
presidency. Elections took place on June 12th. To the surprise of many Mahmoud
Ahmadinajad won with a lanslide victory of well above 60\% of the vote
\cite{TheIranianVote, IranianElectionResultsByProvince}, making this
Ahmadinjad's second term as the President of Iran.


Iranian citizens and many others were outraged with the results of the election.
Laura Secor, from the New York, succinctly sumed up the voice of the people by
stating \begin{quotation} There can be no question that June 12, 2009, Iranian
presidential election was stolen. \cite{TheIranianVote}\end{quotation} Citizens
swiftly began to vocalize their opinions.


Protests began on June 13 and picked up force not only off-line but online as
well. By the fifth day of protest hundreds of thousands of demonstrators took to
the streets of Tehran in opposition of the election's outcome
\cite{IranProtestsFifthDayOfUnrest}. At the same time, ``the Twitterverse
exploded with tweets from people who weren't having it, both in English and in
Farsi.'' \cite{WhyTwitterIsTheMedium}


Iran's government tried to douse out the dissent. In the real world they
arrested, ``at least 500 activitsts, opposition figures, journalists.''
\cite{IranProtestsFifthDayOfUnrest}. The government went even further stating
that the police had orders to shoot anyone appraching the Interior Ministry
\footnote{The Interior Ministry was the part of the government that tallied the
votes of the election} \cite{TheIranianVote}. Online, the government to filter
their internet traffic and for periods of times block social networks like
Twitter \cite{IranBlocksFacebookTwitter}.


While the real world governmental measures taken to stop protests appear to be
more important social networks especially twitter were playing vital roles in
the decemination of information not only for citizens to organize amongst
themselves but for the rest of the world to kno what was happening.

%%twitter's vital importance

%%twitter and other cites were being censored



%%%%%%%%%%%%%%%%%%%%%%%%%
%%% Research Question %%%
%%%%%%%%%%%%%%%%%%%%%%%%%
\section{Research Question} Was it ethical for Iranian protesters, in 2009, to
use Tor to circumvent the network access policies set by the government?


\section{Importance}

%%%%%%%%%%%%%%%%%%%%%%%%%
%%% Extant Arguments from External Sources %%%
%%%%%%%%%%%%%%%%%%%%%%%%%
\section{Extant Arguments} 



\subsection{Arguments For}

%%people don't find tor EVIL

%%3rd parties think the censorship is bad

\subsection{Arguments Against}

%%what tor can enable

%%government believe it is right


%%%%%%%%%%%%%%%%
%%% Analysis %%%
%%%%%%%%%%%%%%%%
\section{Analysis} \begin{itemize} \item Should start with a paragraph showing
    why the SE Code applies to your focus
question.  \item Sub-headings to
    delineate your lines of reasoning are required.  \item All arguments must be
      thoroughly supported by reason and logic.  \item All claims must be
      supported by reputable primary sources and formal data.  \item SE Code
      must be central to the argumentation \begin{itemize} \item You should have
            2-4 distinct sections of the SE code utilized in your analysis
          \begin{itemize} \item If section 1 is central to your argument, it is
                only one of the code sections covered. Do not rely solely on
                section one. Ex: 1.01-1.04 will not suffice for all of your SE
                Code based arguments and citations.  \item If discussion about
                  Òpublic goodÓ is used, there must be data to support it.
                  Simply writing Òit benefits the general public because it
                  would make many people happyÓ is insufficient.  \end{itemize}
            \end{itemize} \item Utilitarian and deontological analysis must be
            present but not be separate sections \item Class reading must be
              referenced as appropriate (at least one paper must be used as the
              basis of one of the arguments).  \item There should be a clear
                cohesiveness to the analysis such that each argument logically
                flows into the next and gently directs the reader toward your
                conclusion while implicitly providing answers to any doubts they
                may have through logic and data.  \item Opinions > dev/null.
              \cite{handout} \end{itemize}

Look at Jason Anderson's how to write a term paper (currently linked as the
paper template) for information on how to write this section.  An example of
possible sections follows \subsection{Why the SE Code Applies}
\subsection{Argument 1} \subsubsection{Code principle 1 that applies}
\subsubsection{Code principle 2 that applies} \subsection{Argument 2}
\subsubsection{Code principle 1 that applies} \subsubsection{Code principle 2
that applies}

\subsubsection*{} Remember to weave the class papers and other ethical systems
arguments in with the se code arguments they shouldn't be separate sections.

%%%%%%%%%%%%%%%%%%
%%% Conclusion %%%
%%%%%%%%%%%%%%%%%%
\section{Conclusion} The conclusion is a summary of your entire anal- ysis. It
should reiterate the answer your audience has been forming while reading your
analysis. New information should never be introduced in your conclusion.
\cite{texTemp}

%end the two column format
\end{multicols} \newpage

%cite all the references from the bibtex you haven't explicitly cited
\nocite{*}

\bibliographystyle{IEEEannot}

\bibliography{texreport}

\end{document}
